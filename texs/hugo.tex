\documentclass[11pt]{article}
\title{T\'el\'egrille }

\usepackage{etex}
\usepackage[french]{babel}
\usepackage[utf8]{inputenc}
\usepackage[T1]{fontenc}
\usepackage[dvipsnames]{xcolor}
\usepackage{amsmath,amssymb}
\usepackage{multicol}
\usepackage[top=1.5cm, bottom=2cm, left=1cm, right=1cm]{geometry}
\usepackage{pstricks-add} % tracé de courbes
\usepackage{enumerate}
\usepackage{fancybox} % Pour les encadrés

\usepackage{tikz}
\frenchbsetup{StandardItemLabels=true} % bullets pour les items
\renewcommand\arraystretch{2}

\newenvironment{changemargin}[2]{\begin{list}{}{%
\setlength{\topsep}{0pt}%
\setlength{\leftmargin}{0pt}%
\setlength{\rightmargin}{0pt}%
\setlength{\listparindent}{\parindent}%
\setlength{\itemindent}{\parindent}%
\setlength{\parsep}{0pt plus 1pt}%
\addtolength{\leftmargin}{#1}%
\addtolength{\rightmargin}{#2}%
}\item }{\end{list} }

\usepackage{ifthen}
\usepackage{intcalc}
\begin{document}
\begin{changemargin}{-.75cm}{0cm}

\begin{center}
\edef\c{21} % Nombre de colonnes
\begin{tikzpicture}[x=.8cm,y=1.2cm]
\foreach[count=\i] \tag in { B, B, B, B, B, B, B, B, B, B, B, B, B, B, B, B, B, B, B, B, B, B, B, B, B, B, B, B, B, B, B, B, B, B, B, B, B, B } {
	\edef\x{\intcalcMod{\i-1}{\c}};
	\edef\y{ -\i/\c+\x/\c+1/\c };
	\ifthenelse{ \equal{\tag}{B} \OR \equal{\tag}{N} }{
		\fill[color=black] (\x,\y) rectangle ({\x+1},{\y-1});
	}{
		\draw[line width=.5pt] (\x,\y) rectangle ({\x+1},{\y-1});
		\draw ({\x+.5},\y) node[below] {\scriptsize\tag};
	}
};
\end{tikzpicture}
\end{center}

\begin{tikzpicture}[x=.8cm,y=1.2cm]
\newcounter{head};
\newcounter{col}[head]
\edef\deltax{0}
\edef\deltay{0}

\foreach[count=\i] \tag in { H, N, N, N, N, N, N, N, N, H, N, N, N, N, N, N, H, N, N, N, N, N, N, N, H, N, N, N, N, N, N, N, N, N, H, N, N, N, N, N, N, N, H, N, N, N, N, N, N, N, N, H, N, N, N, N, N, N, N, N, N, H, N, N, N, N, N, N, N, N, H, N, N, N, N, N, N, N, H, N, N, N, N, N, N, H, N, N, N, N, N, N, N, N, H, N, N, N, N, N, N, N, N, H, N, N, N, N, N, N, N, N, N, H, N, N, N, N, N, N, N, N, H, N, N, N, N, N, N, H, N, N, N, N, N, N, N, H, N, N, N, N, N, N, N, H, N, N, N, N, N, N, N, H, N, N, N, N, N, N, N, H, N, N, N, N, N, N, N, H, N, N, N, N, N, N, N, H, N, N, N, N, N, N, N, N, H, N, N, N, N, N, N, N, N, H, N, N, N, N, N, N, N, H, N, N, N, N, N, N } {

	\ifthenelse{ \equal{\tag}{H} }{
		%Header
		\stepcounter{head}
		\ifthenelse{\value{head}>13}{
			\edef\deltax{12};
			\edef\deltay{13};
		}{}
		\fill[color=Mahogany] (\deltax,{-\value{head}+\deltay}) rectangle ({\deltax+1},{-\value{head}-1+\deltay});
		\draw[color=white] ({\deltax+.5},{-\value{head}-.5+\deltay}) node{\textbf{\Alph{head}
}};
	}{
		\ifthenelse{\value{head}>13}{
			\edef\deltax{12};
			\edef\deltay{13};
		}{}	 % étrange qu'il faille le réécrire...
		\draw[line width=.5pt, fill=white] ({\value{col}+1+\deltax},{-\value{head}+\deltay}) rectangle ({\value{col}+2+\deltax},{-\value{head}-1+\deltay});
		\draw ({\value{col}+1.5+\deltax},{-\value{head}+\deltay}) node[below] {\scriptsize\tag};
		\stepcounter{col};
	}
};


\end{tikzpicture}
\end{changemargin}

\textbf{ A :} Une ancienne région
\textbf{ B :} \textbf{Monstre marin}
\textbf{ C :} Elles se terminent à Ault
\textbf{ D :} Fille de Céphée et Cassiopée
\textbf{ E :} On peut en vivre de drôles
\textbf{ F :} Sortes de crabes
\textbf{ G :} \textit{Le vrai prénom d'une héroïne}
\textbf{ H :} \textbf{Plantes grimpantes}
\textbf{ I :} \textbf{Face à Denise et Philippe}
\textbf{ J :} \textit{Il épouse l'héroïne sus-nommée}
\textbf{ K :} Dans le spectre auditif du chien
\textbf{ L :} Voyageurs
\textbf{ M :} \textit{Un évêque qui offre des chandeliers}
\textbf{ N :} On teste l'intellectuel
\textbf{ O :} Entre haute et basse
\textbf{ P :} Fille de Roderic Borgia
\textbf{ Q :} Elles ne sont pas toutes bonnes à dire
\textbf{ R :} Restent cachés
\textbf{ S :} Une salle vitrée
\textbf{ T :} Réserve nutritive chez les angiospermes
\textbf{ U :} Frêle
\textbf{ V :} Trop de réflexion peut avoir cette conséquence
\textbf{ W :} \textbf{Rue de l'Aurore et du Crépuscule}
\textbf{ X :} Aldohexoses
\textbf{ Y :} Divisions astrologiques

\begin{itemize}
\item Les définitions en gras correspondent à des éléments à découvrir sur le parcours.
\item Les définitions en italique thématique correspondent à des personnages du roman dont le texte est un extrait.
\end{itemize}
\end{document}
